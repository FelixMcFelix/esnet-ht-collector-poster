% !TeX document-id = {0c770b8d-69cd-4601-beb9-6ffc0357e646}
% !TeX program = lualatex
% !BIB program = biber
\documentclass[a0paper, 25pt]{tikzposter}

\usepackage{anyfontsize}

%\usepackage{complexity}
\usepackage{wrapfig}
%\usepackage{microtype}
\usepackage{tikz}
\usetikzlibrary{arrows.meta, calc, 3D, positioning}
%%\usepackage{ifthen}
%%\usepackage{tikz-3dplot}
\usepackage{amssymb}
\usepackage{amsmath}
\usepackage{sfmath}
\usepackage{url}
\usepackage[hidelinks]{hyperref}
\usepackage{cleveref}
\usepackage{enumitem}
\usepackage{booktabs}
%
\usepackage{lmodern}
\usepackage[sfdefault]{FiraSans}
\usepackage{FiraMono}
\renewcommand*\familydefault{\sfdefault}
%\usepackage[T1]{fontenc}
%
\usepackage{graphicx}
\usepackage[export]{adjustbox}
\usepackage{etoolbox}
\usepackage[binary-units, per-mode=symbol]{siunitx}
\robustify\bfseries
\sisetup{detect-all, range-phrase=--, range-units=single, detect-weight=true,detect-inline-weight=math}
%
\usepackage[style=ieee,minnames=1,maxcitenames=2,maxbibnames=2,
mincrossrefs=99,minxrefs=99,
sortcites,
%backend=bibtex,
uniquelist=false]{biblatex}
\addbibresource{papers.bib}
\addbibresource{rfcs.bib}
\addbibresource{confs-journs.bib}
%
%% Addendum formatting (thanks, http://tex.stackexchange.com/questions/339471/indented-addendums-using-biblatex-sourcemaps)
%picky abt et al.
%\usepackage{xpatch}

%\xpatchbibmacro{name:andothers}{%
%	\bibstring{andothers}%
%}{%
%	\bibstring[\emph]{andothers}%
%}{}{}

% emph'd et al. for ieee style
\DefineBibliographyStrings{english}{%
	andothers = {\emph{et al}\adddot}
}
\DeclareFieldFormat[inproceedings]{url}{}
\DeclareFieldFormat[article]{url}{}
\DeclareFieldFormat[inproceedings]{doi}{}
\DeclareFieldFormat[article]{doi}{}
\DeclareFieldFormat[inproceedings]{editor}{}
\DeclareFieldFormat[proceedings]{editor}{}
\DeclareFieldFormat[article]{editor}{}

%\DeclareFieldFormat{crossref}{}
%\AtEveryBibitem{\clearfield{editor}}
\usepackage{xpatch}

\xpatchbibmacro{name:andothers}{%
	\bibstring{andothers}%
}{%
	\bibstring[\emph]{andothers}%
}{}{}

\makeatletter
\newcounter{tablecounter}
\newenvironment{tikztable}[1][]{
	\def \rememberparameter{#1}
	\vspace{10pt}
	\refstepcounter{tablecounter}
	\begin{center}
	}{
		\ifx\rememberparameter\@empty
		\else
		\\[10pt]
		{\small Tab.~\thetablecounter: \rememberparameter}
		\fi
	\end{center}
}
\makeatother

\newcommand{\ini}{\textsuperscript{1}}
\newcommand{\inii}{\textsuperscript{2}}
\title{RTA}
\author{Kyle A. Simpson\ini, Richard Cziva\inii, Yatish Kumar\inii, Chin Guok\inii}
\institute{\ini\emph{University of Glasgow}, \inii\emph{Energy Sciences Network (ESnet)}}
\titlegraphic{
	\resizebox{22.5cm}{!}{
	\begin{tikzpicture}
	\node (lbl) {\includegraphics[keepaspectratio=true,width=\linewidth]{Berkeley_Lab_logo_Vector/Berkeley_Lab_Logo_Small.eps}};
	\node (doe) at ($(lbl.south) + (0,-18.0)$) {\includegraphics[keepaspectratio=true,width=\linewidth]{CMYK_White-Seal_White-Mark_SC_Vertical.eps}};
	\node (esnet) at ($(lbl.east) + (80.0,10.0)$) {\includegraphics[keepaspectratio=true,width=1.8\linewidth]{esnet-logo-min}};
	\node[below = 7cm of esnet] (uofg) {\includegraphics[keepaspectratio=true,width=1.8\linewidth]{UoG_keyline}};
	\end{tikzpicture}
	}
}
%
%\usetitlestyle{Empty}
\settitle{
	\begin{tikzpicture}
        \node (T) [inner sep=0pt] {\begin{minipage}{\linewidth}
                \color{titlefgcolor}
                {\bfseries \Huge \hspace*{10mm}Real-time Performance Analysis of High-Speed,\\\hspace*{10mm}International Science Network Flows \par}
                \vspace*{1em}
                {\Large {\bfseries \hspace{10mm}\@author} \par}
                \vspace*{0.1em}
                {\Large \hspace{10mm}\@institute \par}
                \vspace*{0.1em}
                {\Large\hspace*{10mm}\texttt{\textcolor{white}{\href{mailto:k.simpson.1@research.gla.ac.uk}{k.simpson.1@research.gla.ac.uk}}}}
        \end{minipage}};

        \node at (T.east) [anchor=center, inner sep=0pt, xshift=-12cm] {\@titlegraphic};
    \end{tikzpicture}
}
%
%% University of Glasgow standard colours
\definecolor{uofguniversityblue}{rgb}{0, 0.219608, 0.396078}

\definecolor{uofgheather}{rgb}{0.356863, 0.32549, 0.490196}
\definecolor{uofgaquamarine}{rgb}{0.603922, 0.72549, 0.678431}
\definecolor{uofgslate}{rgb}{0.309804, 0.34902, 0.380392}
\definecolor{uofgrose}{rgb}{0.823529, 0.470588, 0.709804}
\definecolor{uofgmocha}{rgb}{0.709804, 0.564706, 0.47451}

\definecolor{uofglawn}{rgb}{0.517647, 0.741176, 0}
\definecolor{uofgcobalt}{rgb}{0, 0.615686, 0.92549}
\definecolor{uofgturquoise}{rgb}{0, 0.709804, 0.819608}
\definecolor{uofgsunshine}{rgb}{1.0, 0.862745, 0.211765}
\definecolor{uofgpumpkin}{rgb}{1.0, 0.72549, 0.282353}
\definecolor{uofgthistle}{rgb}{0.584314, 0.070588, 0.447059}
\definecolor{uofgpillarbox}{rgb}{0.701961, 0.047059, 0}
\definecolor{uofglavendar}{rgb}{0.356863, 0.301961, 0.580392}

\definecolor{uofgsandstone}{rgb}{0.321569, 0.278431, 0.231373}
\definecolor{uofgforest}{rgb}{0, 0.317647, 0.2}
\definecolor{uofgburgundy}{rgb}{0.490196, 0.133333, 0.223529}
\definecolor{uofgrust}{rgb}{0.603922, 0.227451, 0.023529}

% Mix sandstone and rust for great rust-ice
\colorlet{uofghybrid}{uofgsandstone!70!uofgrust}

\definecolor{lblblue}{RGB}{2, 46, 77}
%
%\hypersetup{
%	colorlinks,
%	citecolor=uofgpumpkin!85!uofgpillarbox,
%	filecolor=black,
%	linkcolor=black,
%	urlcolor=white
%}
%
\definecolorstyle{LBL}{
}{
    % Background Colors
    \colorlet{backgroundcolor}{white}%uofgmocha!95!yellow}%uofgsandstone!80!white}
    \colorlet{framecolor}{lblblue}
    % Title Colors
    \colorlet{titlefgcolor}{white}
    \colorlet{titlebgcolor}{lblblue}
    % Block Colors
    \colorlet{blocktitlebgcolor}{lblblue}
    \colorlet{blocktitlefgcolor}{white}
    \colorlet{blockbodybgcolor}{white}
    \colorlet{blockbodyfgcolor}{black}
    % Innerblock Colors
    \colorlet{innerblocktitlebgcolor}{uofghybrid}%rust}%uofguniversityblue}
    \colorlet{innerblocktitlefgcolor}{black}
    \colorlet{innerblockbodybgcolor}{uofgsandstone}
    \colorlet{innerblockbodyfgcolor}{black}
    % Note colors
    \colorlet{notefgcolor}{black}
    \colorlet{notebgcolor}{uofgrust}
    \colorlet{noteframecolor}{red}
}
%
\usetheme{Autumn}
\usecolorstyle{LBL}
%
\tikzposterlatexaffectionproofoff
%
\useblockstyle[bodyverticalshift=0cm, roundedcorners=0]{Default}
%
%\renewcommand{\Huge}{\fontsize{96}{124}\selectfont}
%
%% Styles for drawings
%
%\tikzset{edge/.style={line width=3pt, color=uofgsandstone}}
%\tikzset{ledge/.style={line width=3pt, color=uofgsandstone!40!white}}
%\tikzset{hedge/.style={line width=3pt, color=uofgsandstone, dashed}}
%
%\setlength\intextsep{0pt}

\begin{document}
\maketitle
\begin{columns}
	\column{0.33}
	{
	\colorlet{blocktitlefgcolor}{white}
	\colorlet{blockbodybgcolor}{uofgcobalt!90!lblblue}
	\colorlet{blocktitlebgcolor}{uofgcobalt!90!lblblue}
	\colorlet{blockbodyfgcolor}{white}
	\block
	[bodyverticalshift=0cm, bodyinnersep=6mm]{}{
		\fontsize{32}{38.4}\selectfont
		\setlength{\parindent}{48pt}
		\noindent
		Science networks carry \textbf{high-speed, critical traffic} from scientific instruments to researchers and laboratories. Instruments at CERN generate \textbf{petabytes of data per second}, which is actively distributed on a global compute and storage infrastructure. The majority of science flows use TCP, with countless variants of congestion control algorithms in use. \textbf{Measuring these flows is a balancing act}: traffic sampling is cheaper, yet packet-based processing is required for tasks such as microburst detection \cite{DBLP:conf/sigcomm/ChenFKRR18}. Collecting telemetry for every packet would improve existing measures and enable new analyses, but presents a difficult engineering challenge.
		
		We present a system to perform \textbf{per-packet, fine-grained monitoring}. We use \textbf{P4 programmable hardware} to implement a telemetry agent that converts packets into a digest with a \textbf{nanosecond-accurate timestamp}. Several of these switches are being deployed in our multi-100Gbit/s network. Digests are analysed by dedicated collector servers, \textbf{engineered to run at wire-rate}.
	}}
	\block[bodywidthscale=0.999]{Architecture}{
		\fontsize{32}{38.4}\selectfont
		\begin{tikzfigure}[Integration of a High Touch collector and flow timestamping at the network edge.\label{fig:ht-arch}]
		\adjincludegraphics[width=\linewidth]{sys-overview-white}
		\vspace{-5cm}
		\end{tikzfigure}
	
		Test text hello.
		
		?? What actual hardware? Agilio CXs
	}
	\column{0.33}
	\block[bodywidthscale=0.9999]{Collectors}{
%	\begin{wrapfigure}[6]{r}{.3\linewidth}
		\fontsize{32}{38.4}\selectfont
		\begin{tikzfigure}[Collector architecture.\label{fig:collector-arch}]
		\resizebox{\linewidth}{!}{\fontsize{10}{12}\selectfont\begin{tikzpicture}
		[stage/.style={draw, rounded rectangle, fill=uofgmocha, align=center},
		pipeline/.style={stage, fill=uofgrose},
		ws/.style={stage, fill=uofgslate, text=white},
		store/.style={stage, fill=uofgsunshine}]
		\node[draw, fill=white!90!uofgpumpkin, minimum width=10.5cm, minimum height=2.7cm] (collectorbox) at (4.6,0) {};
		\node[below right, inner sep=2pt] at (collectorbox.north west) {Collector};
		
		\node[stage](parse) {Parse};
		\node[stage, right = 0.4cm of parse](dispatch) {Dispatch};
		
		\node[pipeline, above right = 0.1cm and 0.7cm of dispatch](ana0) {Analyser 0};
		\node[pipeline, right = 0.3cm of ana0](serialiser0) {Serialiser 0};
		
		\node[pipeline, below right = 0.1cm and 0.7cm of dispatch](anan) {Analyser $n$};
		\node[pipeline, right = 0.3cm of anan](serialisern) {Serialiser $n$};
		
		\node at ($(ana0.south)!0.4!(anan.north)$) {$\vdots$};
		\node at ($(ana0.south)!0.4!(anan.north) + (0.2,-0.1)$) {$n$};
		\node at ($(serialiser0.south)!0.4!(serialisern.north)$) {$\vdots$};
		\node at ($(serialiser0.south)!0.4!(serialisern.north) + (0.2,-0.1)$) {$n$};
		
		\node[store, right = 7.4cm of parse](storage) {Shared\\Storage};
		
		\node[store, below = 1cm of storage] (tsdb) {Time-series\\Database};
		
		\node[ws, above right = 0.2cm and 2cm of storage](ws0) {WS Client 0};
		\node[ws, below right = 0.2cm and 2cm of storage](wsm) {WS Client $m$};
		\node at ($(ws0.south)!0.4!(wsm.north)$) {$\vdots$};
		\node at ($(ws0.south)!0.4!(wsm.north) + (0.25,-0.1)$) {$m$};
		
		\draw[thick, ->] (parse)--(dispatch);
		\draw[thick, ->] (dispatch)--(ana0.west);
		\draw[thick, ->] (dispatch)--(anan.west);
		
		\draw[thick, ->] (ana0)--(serialiser0);
		\draw[thick, ->] (anan)--(serialisern);
		
		\draw[thick, ->] (serialiser0)--(storage);
		\draw[thick, ->] (serialisern)--(storage);
		\draw[dashed, ->] (serialiser0)--(tsdb);
		\draw[dashed, ->] (serialisern)--(tsdb);
		
		\draw[<->] (storage)--(ws0.west);
		\draw[<->] (storage)--(wsm.west);
		
		\node[above = 1.3cm of parse] (locallabel) {Local};
		\node[right = 10.3cm of locallabel] (remotelabel) {Remote};
		
		\draw[-, dotted] ($(remotelabel) - (1,-0.5)$)--($(remotelabel) - (1,4.5)$);
		\end{tikzpicture}}
		\end{tikzfigure}

		Collectors sure do work! Here are some of the algorithms we've implemented...
		\begin{itemize}
			\item Rate Monitoring (point estimate and sliding window).
			\item Packet loss and retransmission detection.
			\item Online SRTT estimation via Karn's algorithm \cite{DBLP:journals/ccr/KarnP87}.
			\item Bytes-in-flight tracking.
			\item Congestion window size estimation via \citeauthor{DBLP:conf/sosr/GhasemiBR17}'s algorithm \cite{DBLP:conf/sosr/GhasemiBR17}.
		\end{itemize}
	
		And here's how fast it goes...
		
		?? Convert into actual useful numbers -- what is the peak between all flows? What is the peak per flow?
		
		?? Where do the numbers come from? Smaller-scale machine, so promising.
		
		?? Preliminary test with Rust suggest higher throughput.
		
	
	\begin{tikztable}[Pipeline processing rates.\label{tab:rates}]
		\resizebox{\linewidth}{!}{\begin{tabular}{@{}ccccc@{}}
				\toprule\multicolumn{1}{c}{Metric} & \multicolumn{1}{c}{Parse} & \multicolumn{1}{c}{Dispatch} & \multicolumn{1}{c}{Analysis} & \multicolumn{1}{c}{Serialisation} \\ \midrule
				Packet rate (kpps) & 1300 & 775 & \numrange{310}{485} & 675 \\
				Flow rate, jumbo frames (\si{\giga\bit\per\second}) & 93.6 & 55.8 & \numrange{22.3}{34.9} & 48.6 \\
				Flow rate, \SI{1500}{\byte} (\si{\giga\bit\per\second}) & 15.6 & 9.3 & \numrange{3.7}{5.8} & 8.1 \\
				\bottomrule
			\end{tabular}
		}
	\end{tikztable}

%	\alert{Does this work?}
%	\end{wrapfigure}
}
	\column{0.33}
	\block{Next Collectors}{
		\fontsize{32}{38.4}\selectfont
		Stateful TCP analysis of this kind acts as a pilot project for the feasibility of the system. We intend to explore other collector designs, including:
		\begin{itemize}
			\item Peak utilisation tracking (i.e., at \SI{1}{\second} granularity).
			\item Monitoring microbursts in the network via timing information.
			\item Visualisation/measurement of queueing.
		\end{itemize}
	}
	\block{Datasets}{
		\fontsize{32}{38.4}\selectfont
		We provide data about some things...
		\begin{itemize}
			\item A
			\item B
			\item C
		\end{itemize}
		We've made this available at: \\ \textbf{\url{https://bit.ly/esnet-ht-data}}.
	}
	\block{Future Work}{
		\fontsize{32}{38.4}\selectfont
		Synchronisation?
	}
	\block{References}{\printbibliography[heading=none]}
\end{columns}
\end{document}
%\maketitle